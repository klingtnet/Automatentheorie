\documentclass{scrartcl}
\addtokomafont{disposition}{\rmfamily}

\usepackage{scrpage2}
\usepackage{xltxtra}
\usepackage{polyglossia}
\usepackage{lastpage}
\usepackage{amsmath, amssymb}
\usepackage{stmaryrd}
\usepackage{caption}
\usepackage{csquotes}
\usepackage{tikz}
\usepackage{fontspec}
% \setmainfont{Linux Libertine O}
\usetikzlibrary{arrows,automata}
\usetikzlibrary{trees}

\setmainlanguage{german}

\setlength{\parskip}{\baselineskip}
\setlength{\parindent}{0pt}

\pagestyle{scrheadings}
\clearscrheadfoot
\ihead[]{Andreas Linz, Lucas Stadler}
\chead[]{\textsc{Automatentheorie}}
\ohead[]{\today}
\cfoot[\pagemark]{\textnormal{\pagemark{} / \pageref{LastPage}}}
\setheadsepline{1pt}

\begin{document}

\section{Übung 04}

\subsection{H 4-1}

\begin{itemize}

\item[a)] \begin{align*}
g|_{A^0} &= \left\{(1,z)\;|\;1\in A, z = f(x) = 1 \in M\right\}\\
g|_{A^1} &= \left\{(x,z)\;|\;x\in A \wedge f(x) = z \in M\right\}\\
g|_{A^2} &= \left\{(xy,z)\;|\;x,y\in A \wedge z = f(x)\cdot f(y)\right\}\\
         &\vdots\\
g|_{A^{n+1}} &= \left\{(u\cdot v,z)\;|\;u\in A^n, v \in A \wedge z = g|_{A^n}(u) \cdot f(v)\right\}
\end{align*}
$g$ ist ein Morphismus weil es die folgen Eigenschaften erfüllt:
\begin{itemize}
  \item das Einselement wird abgebildet durch $g|_{A^0}$
  \item $\forall x,y \in A : g(x\cdot y) = g(x) \cdot g(y)$
\end{itemize}
\item[b)] Sei $f : A \mapsto M$ eine surjektive Abbildung. Nach \textit{Theorem 2.4a} gibt es einen Morphismus $g : A^* \mapsto M$ mit $g|_A = f$. Da $f$ bereits surjektiv ist, so ist $g$ trivialerweise auch surjektiv und somit ein Epimorphismus.

\end{itemize}
\newpage

\subsection{H 4-2}

\begin{figure}[h!]
\centering
\begin{tikzpicture}[node distance=3cm, auto, >=stealth]
  \node[state, initial by arrow, initial text={}] (q_0) {0};
  \node[state,accepting by arrow] (q_2) [right of=q_0] {2};
  \node[state,accepting by arrow] (q_1) [above of=q_2] {1};
  \node[state,accepting by arrow] (q_3) [below of=q_2] {3};

  \path[->] (q_0) edge [loop above] node {0,1,2} (q_0)
            (q_0) edge node [near start] {2} (q_1)
            (q_1) edge [loop above] node {0,1,2} (q_1)
            (q_0) edge node [near start] {3} (q_2)
            (q_2) edge [loop above] node {0,1,2,3} (q_2)
            (q_0) edge node [near start] {6} (q_3)
            (q_3) edge [loop above] node {0,1,2,4,5,6} (q_3)
  ;
\end{tikzpicture}
\caption*{a)}
\end{figure}

\begin{figure}[h!]
\centering
\begin{tikzpicture}[node distance=3cm, auto, >=stealth]
  \node[state, initial by arrow, initial text={}] (q_0) {0};
  \node[state,accepting by arrow] (q_1) [right of=q_0] {1};

  \path[->] (q_0) edge [loop above] node {$\{0\} \cup E$} (q_0)
            (q_0) edge [bend left]  node {$U$} (q_1)
            (q_1) edge [bend left] node {$U$} (q_0)
            (q_1) edge [loop above] node {$\{0\} \cup E$} (q_1)
  ;
\end{tikzpicture}
\caption*{b)}
\end{figure}

\subsection{H 4-3}

\begin{itemize}
\item \textbf{(a)}: $L = \{a, aaa\}^* = \{\epsilon, a, aa, aaa, aaaa, \ldots\} = \{a\}^*$:

  $\{a\}^*$ hat lediglich eine Äquivalenzklasse: $[a] = a^*$, demzufolge gilt dasselbe auch für $\{a, aaa\}^*$. Das syntaktische Monoid besteht demnach auch nur aus dieser einen Äquivalenzklasse.

\item \textbf{(b)}: $L = \{ba\}^*$ im Monoid $\{a, b\}^*$ hat folgendes syntaktisches Monoid:
  \begin{equation*}
    \{[a] = a(ba)^*, [b] = (ba)^*b, [ab] = (ab)^*, [ba] = (ba)^*\}
  \end{equation*}

\item \textbf{(c)}: $L = \{2, 3, 6\}$ im Monoid $(\mathbb{N}, \text{max})$:
  \begin{align*}
    [1] &= \{0, 1\}\\
    [2] &= [3] = \{2, 3\}\\
    [6] &= \{4, 5, 6\}\\
    [42] &= \{x \in \mathbb{N} : x > 6\}
  \end{align*}

\item \textbf{(d)}: $L = \{7\}$ im Monoid $(\mathbb{Z}, +)$:

  Zu $L$ gibt es unendlich viele Äquivalenzklassen: $[1] = \{6\}, [2] = \{5\}, \ldots$, welche alle die Form $[x] = y$ mit $x + y = 7$ haben.

  Demnach ist das syntaktische Monoid $M/\sim_\{7\}$ ebenfalls unendlich:
  \begin{equation*}
    \{[x] = y \text{ mit } y \in \mathbb{Z} \wedge x + y = 7\}
  \end{equation*}

\item \textbf{(e)}: $L = \{(n, n) : n \in \mathbb{N}\}$ im Monoid $(\mathbb{N}, +)^2$:

  Für diese Sprache gibt es ebenfalls unendlich viele Äquivalenzklassen:
  \begin{equation*}
    [(x, y)] = \{(u, v) \in \mathbb{N}^2 \text{ mit } x + u = y + u\}
  \end{equation*}

  Das syntaktische Monoid enthält alle Äquivalenzklassen dieser Form.
\end{itemize}

\end{document}
