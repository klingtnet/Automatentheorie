\documentclass{scrartcl}
\addtokomafont{disposition}{\rmfamily}

\usepackage{scrpage2}
\usepackage{xltxtra}
\usepackage{polyglossia}
\usepackage{lastpage}
\usepackage{amsmath, amssymb}
\usepackage{stmaryrd}
\usepackage{caption}
\usepackage{csquotes}
\usepackage{tikz}
\usepackage{fontspec}
\usepackage{enumerate}
% \setmainfont{Linux Libertine O}
\usetikzlibrary{arrows,automata}
\usetikzlibrary{trees}

\setmainlanguage{german}

\setlength{\parskip}{\baselineskip}
\setlength{\parindent}{0pt}

\pagestyle{scrheadings}
\clearscrheadfoot
\ihead[]{Andreas Linz, Lucas Stadler}
\chead[]{\textsc{Automatentheorie}}
\ohead[]{\today}
\cfoot[\pagemark]{\textnormal{\pagemark{} / \pageref{LastPage}}}
\setheadsepline{1pt}

\begin{document}

\section{Übung 06}

\subsection{H 6-1: \textsl{Welche der Monoide sind aperiodisch?}}

\begin{enumerate}[a)]
  \item $\left(\mathbb{Z}, +\right)$, nicht aperiodisch, da
    $\forall x \in \mathbb{Z}, x \neq 0 : \left|x^n\right| < \left|x^{n+1}\right| = \left|x^n\right| + \left|x\right|$.
  \item $\left(\mathbb{N}, min\right)$ ist aperiodisch, da wiederholte Anwendung von \textit{min} auf denselben Wert das gleiche Ergebnis liefern, also:
  $$x^n = min(\underbrace{x, \ldots, x}_{\text{n-mal}}) = x$$
  \newcommand{\cn}{\cdot_k}
  \item $\left(\left\{0,1,2,\ldots,k\right\}\cn \right)$
  $$x\cn y =
    \begin{cases}
      k           & x \cdot y > k\\
      x \cdot y   & sonst
    \end{cases}
  $$
  Für die Aperiodizität ist nur der Fall $2^n = 2^{n+1}$ relevant, da $0^n = 0^{n+1} = 0$,
  $1^n = 1^{n+1} = 1$ und wenn $x^n = x^{n+1}$ für $x = 2$ gilt, dann gilt es auch für
  alle $x > 2$.

  Das kleinste $n$ für das $x^n = x^{n+1}$ gilt ist also $n = \left\lceil \log_2 k \right\rceil$:
  \begin{align*}
    2^n &= 2^{\left\lceil \log_2 k \right\rceil} = k\\
    2^{n+1} &= 2^n \cdot 2 = 2^{\left\lceil \log_2 k \right\rceil} \cdot 2 = k \cn 2 = k\\
    &\Rightarrow 2^n = 2^{n+1} \text{ für } n = \left\lceil \log_2 k \right\rceil
  \end{align*}
  \item Das Monoid $\left(\left\{1,A,B\right\}, \diamond\right)$ ist aperiodisch für alle $n \geq 1$, da:
  $$
    \forall x \in \{1,A,B\} : x^1 = x \diamond x = x\\
  $$
  Demzufolge gilt auch:
  \begin{align*}
    x^n &= \underbrace{x \diamond \ldots \diamond x}_{\text{n-mal}} = x\\
    x^{n+1} &= x^n \diamond x = x
  \end{align*}
\end{enumerate}

\subsection{H 6-2: \textsl{Index für sternfreie aperiodische Sprachen}}

Def 3-2 (a): L ist aperiodisch $\Leftrightarrow \exists n \in \mathbb{N}\wedge\forall x,y,z \in M: xy^nz \Leftrightarrow xy^{n+1}z \in L$

\begin{enumerate}[(a)]
  \item
    \begin{align*}
      L &= a^+c^+\\
      i(L) &= 2
    \end{align*}
    Für $i = 1$ ist $x=a, y=ac, z=c \in L$ aber $xy^{i+1}z = a\cdot acac\cdot c$ nicht.
  \item
    \begin{align*}
      L &= (abc)^*\\
      i(L) &= 2
    \end{align*}
  \item
    \begin{align*}
    L &= \left\{ w : |w|_a \leq 3\right\}\\
    i(L) &= 4
    \end{align*}
  \item
    \begin{align*}
    L &= \left\{ w : |w|_{aba} = 2\right\}\\
    i(L) &= 3
    \end{align*}
\end{enumerate}

\subsection{H 6-3: \textsl{Aperiodizität syntaktischer Monoide}}

\begin{align*}
  A &= \left\{a,b,c\right\}\\
  L &= a^+c^+\\
  [a] &= \{a\}\\
  [c] &= \{c\}\\
  [ac] &= \left\{a(ac^m)c\;|\;m \geq 2\right\}\\
  [b] &= A^*caA^* \cup A^*bA^* \cup a^+ \cup c^+
\end{align*}

Nach Def. 3-2 (b) ab $i(L) = 2$ aperiodisch, siehe H 6-2.

%Das syntaktische Monoid $M/_{\~{}L}$ besteht aus den oben genannten Äquivalenzklassen.

% \begin{enumerate}[(a)]
%   \item $L = a^+c^+$
%   \item $L = (abc)^*$
%     \begin{minipage}[t]{0.3\textwidth}
%       \begin{align*}
%         [a] &= (abc)^*a\\
%         [b] &= b(cab)^*\\
%         [c] &= c(abc)^*\\
%       \end{align*}
%     \end{minipage}
%     \begin{minipage}[t]{0.3\textwidth}
%       \begin{align*}
%       [ab] &= ab(cab)^*\\
%       [ac] &= a(bca)^*c\\
%       [ba] &= b(abc)^*a\\
%       [bc] &= b(cab)^*c\\
%       [ca] &= c(abc)^*a\\
%       [cb] &= c(abca)^*b
%       \end{align*}
%     \end{minipage}
%     \begin{minipage}[t]{0.3\textwidth}
%       \begin{align*}
%       [abc] &= (abc)^*\\
%       [bca] &= (bca)^*\\
%       [cab] &= (cab)^*
%       \end{align*}
%     \end{minipage}
%   \item $L = \left\{ w : |w|_a \leq 3\right\}$
%   \item $L = \left\{ w : |w|_{aba} = 2\right\}$
% \end{enumerate}

\end{document}
