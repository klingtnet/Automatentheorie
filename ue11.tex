\documentclass{scrartcl}
\addtokomafont{disposition}{\rmfamily}

\usepackage{scrpage2}
\usepackage{xltxtra}
\usepackage{polyglossia}
\usepackage{lastpage}
\usepackage{amsmath, amssymb}
\usepackage{stmaryrd}
\usepackage{caption}
\usepackage{csquotes}
\usepackage{tikz}
\usepackage{fontspec}
\usepackage{enumerate}
\usepackage{mathtools} % coloneqq
% \setmainfont{Linux Libertine O}

\setmainlanguage{german}

\setlength{\parskip}{\baselineskip}
\setlength{\parindent}{0pt}

\pagestyle{scrheadings}
\clearscrheadfoot
\ihead[]{Andreas Linz, Lucas Stadler}
\chead[]{\textsc{Automatentheorie}}
\ohead[]{\today}
\cfoot[\pagemark]{\textnormal{\pagemark{} / \pageref{LastPage}}}
\setheadsepline{1pt}

\begin{document}

\section{Übung 11}

\subsection{H 11-1}

\textsl{Sei $A = \{a, b\}$ und $S$ ein beliebiger Semiring. Berechnen Sie jeweils die Reihen $s = ((1_S a + 1_S b)^*)^2$ und $t = ((1_S a + 1_S b)^2)^*$ für die folgenden Semiringe $S$:}

\begin{enumerate}[(a)]
    \item $S = (\mathbb{N}, +, \cdot, 0, 1)$
    \item $S = (\mathbb{Z} \cup \{-\infty\}, \text{max}, +, -\infty, 0)$
\end{enumerate}

\subsection{H 11-2}

\textsl{Sei $A$ ein Alphabet und $(\mathbb{R}, +, \cdot, 0, 1)$ der Körper der reelen Zahlen.}

\begin{enumerate}[(a)]
    \item \textsl{Zeigen Sie die folgende Aussage: Ist $s \in R\langle\langle A^* \rangle\rangle$ erkennbar, dann gibt es ein $K > 0$ mit $|(s, w)| \leq K^{|w|+2}$ für alle $w \in A^*$.}
    \item \textsl{Geben Sie eine Reihe $s \in R\langle\langle A^* \rangle\rangle$ an, welche nicht erkennbar ist. Begründen Sie Ihre Wahl!}
\end{enumerate}

\subsection{H 11-3}

\textsl{Zeigen Sie, dass die Reihe $s \in \mathbb{N}\langle\langle A^* \rangle\rangle$, welche durch $(s, a^n) = f_n$ definiert ist, rational ist. Finden Sie außerdem einen gewichteten Automaten über $\mathbb{N}$, welcher $s$ erkennt.}

\end{document}
