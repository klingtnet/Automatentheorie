\documentclass{scrartcl}
\addtokomafont{disposition}{\rmfamily}

\usepackage{scrpage2}
\usepackage{xltxtra}
\usepackage{polyglossia}
\usepackage{lastpage}
\usepackage{amsmath, amssymb}
\usepackage{stmaryrd}
\usepackage{caption}
\usepackage{csquotes}
\usepackage{tikz}
\usepackage{fontspec}
\usepackage{enumerate}
\usepackage{mathtools} % coloneqq
% \setmainfont{Linux Libertine O}

\setmainlanguage{german}

\setlength{\parskip}{\baselineskip}
\setlength{\parindent}{0pt}

\pagestyle{scrheadings}
\clearscrheadfoot
\ihead[]{Andreas Linz, Lucas Stadler}
\chead[]{\textsc{Automatentheorie}}
\ohead[]{\today}
\cfoot[\pagemark]{\textnormal{\pagemark{} / \pageref{LastPage}}}
\setheadsepline{1pt}

\begin{document}

\section{Übung 11}

\subsection{H 11-1}

\textsl{Sei $A = \{a, b\}$ und $S$ ein beliebiger Semiring. Berechnen Sie jeweils die Reihen $s = ((1_S a + 1_S b)^*)^2$ und $t = ((1_S a + 1_S b)^2)^*$ für die folgenden Semiringe $S$:}

\begin{enumerate}[(a)]
    \item $S = (\mathbb{N}, +, \cdot, 0, 1)$

      \begin{align*}
        (1_S ab, w) &= (1_S a + 1_S b, w) = (1_S a, w) + (1_S b, w)\\
          &= \begin{cases}
            1_S, &\text{wenn } w = a \text{ oder } w = b\\
            0,   &\text{sonst}
          \end{cases}
          \\
        \\
        ((1_S ab)^n, w) &=
          \sum_{w = u_1 \cdots u_n} (1_S ab, u_1) \cdots (1_S ab, u_n)\\
          &= (1_S ab, u_1) \cdots (1_S ab, u_n) \text{ mit } u_1, \ldots, u_n \in A\\
          &= \begin{cases}
            1, &\text{wenn } |w| = n\\
            0, &\text{sonst}
          \end{cases}
          \\
        \\
        ((1_S ab)^*, w) &= \sum_{n \geq 0} ((1_S ab)^n, w)\\
          &= \underbrace{\sum_{0 < n < |w|} ((1_S ab)^n, w)}_{0}
            + ((1_S ab)^{|w|}, w)
            + \underbrace{\sum_{n > |w|} ((1_S ab)^n, w)}_{0}\\
          &= ((1_S ab)^{|w|}, w)\\
          &= (1w, w)\\
        \\
        (((1_S ab)^*)^2, w) &= \sum_{w = uv} ((1_S ab)^*, u) \cdot ((1_S ab)^*, v)\\
          &= |\{w = uv\}|\\
        \displaybreak\\
        (((1_S ab)^2)^n, w) &= \sum_{w = u_1 \cdots u_n} ((1_S ab)^2, u_1) \cdots ((1_s ab)^2, u_n)\\
          &= \begin{cases}
            1, &\text{wenn } |w| \bmod 2 = 0\\
            0, &\text{sonst}
          \end{cases}
          \\
        %\\
        %(((1_S ab)^2)^*, w) &= \sum_{n \geq 0} ((1_S ab)^2)^n\\
        %% Sollte insgesamt \infty ergeben? (Oder 0, falls |w| \bmod 2 \neq 0.)
      \end{align*}

    \item $S = (\mathbb{Z} \cup \{-\infty\}, \text{max}, +, -\infty, 0)$
    \begin{align*}
    (0\,ab, w) &=
      \big(\max(0\,a, 0\,b), w\big) = \max\big((0\,a, w), (0\,b, w)\big)\\
      &=\begin{cases}
        0, &w = a \lor w = b\\
        -\infty, &\text{sonst}
      \end{cases}\\
    \\
    \big((0\,ab)^n, w\big)
      &=\max\big(\overbrace{(0\,ab, u_1) \ldots (0\,ab, u_n)}^{w = u_1 \ldots u_n}\big)\\
      &=\begin{cases}
        0, &|w| = n\\
        -\infty, &\text{sonst}
      \end{cases}\\
    \\
    \big((0\,ab)^*, w\big)
      &= \max\big(\overbrace{((0\,ab)^0, w), ((0\,ab)^1, w), \ldots{}}^{=-\infty}, ((0\,ab)^{n=|w|}, w), \overbrace{((0\,ab)^{n+1}, w), ((0\,ab)^{n+2}, w), \ldots{}}^{=-\infty}\big)\\
      &= \big((0\,w), w\big)\\
    \\
    \big(((0\,ab)^*)^2, w\big)
      &= \max_{w=uv}\Big(\big((0\, ab)^*, u\big)+\big((0\, ab)^*, v\big)\Big)\\
      &= \begin{cases}
        0, &w=uv \land |u|>0 \land |v|>0\\
        -\infty\text{sonst}
      \end{cases}
    \\
    \big(((0\,ab)^2)^*, w\big)
      &= ((0\,ab)^2, w) =
    \end{align*}
\end{enumerate}

\subsection{H 11-2}

\textsl{Sei $A$ ein Alphabet und $(\mathbb{R}, +, \cdot, 0, 1)$ der Körper der reelen Zahlen.}

\begin{enumerate}[(a)]
    \item \textsl{Zeigen Sie die folgende Aussage: Ist $s \in R\langle\langle A^* \rangle\rangle$ erkennbar, dann gibt es ein $K > 0$ mit $|(s, w)| \leq K^{|w|+2}$ für alle $w \in A^*$.}
    \item \textsl{Geben Sie eine Reihe $s \in R\langle\langle A^* \rangle\rangle$ an, welche nicht erkennbar ist. Begründen Sie Ihre Wahl!}
\end{enumerate}

\subsection{H 11-3}

\textsl{Zeigen Sie, dass die Reihe $s \in \mathbb{N}\langle\langle A^* \rangle\rangle$, welche durch $(s, a^n) = f_n$ definiert ist, rational ist. Finden Sie außerdem einen gewichteten Automaten über $\mathbb{N}$, welcher $s$ erkennt.}

\begin{align*}
  f_n = (s, a^n) &= \begin{cases}
    0, &n = 0\\% \lor w = \varepsilon\\
    1, &n = 1\\% \lor w = a\\
    (s, a^{n-1}) + (s, a^{n-2}), &\text{sonst}
  \end{cases}
\end{align*}


\end{document}
