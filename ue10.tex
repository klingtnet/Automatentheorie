\documentclass{scrartcl}
\addtokomafont{disposition}{\rmfamily}

\usepackage{scrpage2}
\usepackage{xltxtra}
\usepackage{polyglossia}
\usepackage{lastpage}
\usepackage{amsmath, amssymb}
\usepackage{stmaryrd}
\usepackage{caption}
\usepackage{csquotes}
\usepackage{tikz}
\usepackage{fontspec}
\usepackage{enumerate}
\usepackage{mathtools} % coloneqq
% \setmainfont{Linux Libertine O}
\usetikzlibrary{arrows,automata, graphs}
\usetikzlibrary{trees}

\setmainlanguage{german}

\setlength{\parskip}{\baselineskip}
\setlength{\parindent}{0pt}

\pagestyle{scrheadings}
\clearscrheadfoot
\ihead[]{Andreas Linz, Lucas Stadler}
\chead[]{\textsc{Automatentheorie}}
\ohead[]{\today}
\cfoot[\pagemark]{\textnormal{\pagemark{} / \pageref{LastPage}}}
\setheadsepline{1pt}

\begin{document}

\section{Übung 10}

\subsection{H 10-1}

\textsl{Sei $A$ ein Alphabet, $S$ ein bel. Semiring und $s: A^* \rightarrow S$ eine erkennbare Funktion. Weiterhin sei $\# \in A$ ein neues Zeichen und die Funktion $t: (A \cup \{\#\})^* \rightarrow S$ definiert \ldots}

Nach \emph{Theorem 5.6} hat $s$ eine Repräsentation $s(w) = \lambda\cdot\mu(w)\cdot\gamma$. Damit kann eine Repräsentation für t erzeugt werden:

\begin{align*}
    t(w) &=  \lambda\cdot\mu'(w)\cdot\gamma\\
    \mu'(w) &= \begin{cases}
        \mu(v)  &\text{falls}\; w = u\#v \land v \in A^*\\
        \mu(w)  &\text{sonst}
    \end{cases}
\end{align*}

Da $t$ die Repräsentation $(\lambda, \mu', \gamma)$ hat, ist $t$ ebenfalls nach \emph{Theorem 5.6} erkennbar.

\subsection{H 10-2}

\textsl{Sei $A = \{a,b,c\}$. Geben Sie gewichtete Automaten an, welche die folgenden Funktionen $s: A^* \rightarrow S$ über dem angegebenen Semiringen $S$ erkennen.}

\begin{enumerate}[(a)]
    \item $S = (\mathbb{Z}, +, \cdot, 0, 1)$, mit $s(w) = 2|w|_a -3|w|_b + 4|w|_c$

      \begin{figure}[ht]
        \centering
        \begin{tikzpicture}[node distance=2.25cm, auto, >=stealth]
          \node[initial by arrow, state, initial text={}] (q_0) {$q_0$};
          \node[state, accepting by arrow] (q_2) [right of=q_0] {$q_2$};
          \node[state, accepting by arrow] (q_1) [right of=q_0, above of=q_0] {$q_1$};
          \node[state, accepting by arrow] (q_3) [right of=q_0, below of=q_0] {$q_3$};

          \path[->] (q_0) edge node [above, near start, yshift=2.5mm] {$a/_2$}  (q_1);
          \path[->] (q_0) edge node {$b/_{-3}$} (q_2);
          \path[->] (q_0) edge node [below, near start, yshift=-2.5mm] {$c/_4$}  (q_3);

          \path[->] (q_0) edge [loop above] node [left] {$*/_1$} (q_0);
          \path[->] (q_1) edge [loop above] node [left] {$*/_1$} (q_1);
          \path[->] (q_2) edge [loop above] node [left] {$*/_1$} (q_2);
          \path[->] (q_3) edge [loop above] node [left] {$*/_1$} (q_3);
        \end{tikzpicture}
      \end{figure}

    \item $S = (\mathcal{P}(A^*), \cup, \cdot, \varnothing, \{\varepsilon\})$ mit
      \begin{equation*}
        s(w) = \begin{cases}
          \{a^n\}     &\text{falls}\;w = a^nbv\;\text{mit}\;v \in A^*\\
          \{a^2n\}    &\text{falls}\;w=a^ncv\;\text{mit}\;v \in A^*\\
          \varnothing &\text{sonst}
        \end{cases}
      \end{equation*}

      \begin{figure}[ht]
        \centering
        \begin{tikzpicture}[node distance=1.75cm, auto, >=stealth]
          \node[initial by arrow, state, initial text={}] (q_0) {$q_0$};

          \node[state] (q_1) [right of=q_0, above of=q_0]{$q_1$};
          \node[state, accepting by arrow] (q_3) [right of=q_1] {$q_3$};

          \node[state] (q_2) [right of=q_0, below of=q_0]{$q_2$};
          \node[state, accepting by arrow] (q_4) [right of=q_2] {$q_4$};

          \path[->] (q_0) edge node {$a/_{\{a\}}$} (q_1);
          \path[->] (q_1) edge [loop above] node {$a/_{\{a\}}$} (q_1);
          \path[->] (q_1) edge node {$b/_{\{\epsilon\}}$} (q_3);
          \path[->] (q_3) edge [loop above] node {$*/_{\{\epsilon\}}$} (q_3);

          \path[->] (q_0) edge node [below, xshift=-5mm] {$a/_{\{aa\}}$} (q_2);
          \path[->] (q_2) edge [loop above] node {$a/_{\{aa\}}$} (q_2);
          \path[->] (q_2) edge node {$c/_{\{\epsilon\}}$} (q_4);
          \path[->] (q_4) edge [loop above] node {$*/_{\{\epsilon\}}$} (q_4);
        \end{tikzpicture}
      \end{figure}
\end{enumerate}

\subsection{H 10-3}

\textsl{Sei $S$ ein kommutativer Semiring und $s: A^* \rightarrow S$ eine erkennbare Funktion. Für ein Wort $w = w_1 \ldots w_n \in A^*$ definieren wir $\tilde{w} = w_n \ldots w_1$. Zeigen Sie, dass die Funktion $\tilde{s} : A^* \rightarrow S$ mit $\tilde{s}(w) = s(\tilde{w})$ wieder erkennbar ist.}

\begin{align*}
    s(\tilde{w}) = \lambda\mu
\end{align*}

\end{document}
