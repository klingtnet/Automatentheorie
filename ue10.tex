\documentclass{scrartcl}
\addtokomafont{disposition}{\rmfamily}

\usepackage{scrpage2}
\usepackage{xltxtra}
\usepackage{polyglossia}
\usepackage{lastpage}
\usepackage{amsmath, amssymb}
\usepackage{stmaryrd}
\usepackage{caption}
\usepackage{csquotes}
\usepackage{tikz}
\usepackage{fontspec}
\usepackage{enumerate}
\usepackage{mathtools} % coloneqq
% \setmainfont{Linux Libertine O}
\usetikzlibrary{arrows,automata, graphs}
\usetikzlibrary{trees}

\setmainlanguage{german}

\setlength{\parskip}{\baselineskip}
\setlength{\parindent}{0pt}

\pagestyle{scrheadings}
\clearscrheadfoot
\ihead[]{Andreas Linz, Lucas Stadler}
\chead[]{\textsc{Automatentheorie}}
\ohead[]{\today}
\cfoot[\pagemark]{\textnormal{\pagemark{} / \pageref{LastPage}}}
\setheadsepline{1pt}

\begin{document}

\section{Übung 10}

\subsection{H 10-1}

\textsl{Sei $A$ ein Alphabet, $S$ ein bel. Semiring und $s: A^* \rightarrow S$ eine erkennbare Funktion. Weiterhin sei $\# \in A$ ein neues Zeichen und die Funktion $t: (A \cup \{\#\})^* \rightarrow S$ definiert \ldots}

Nach \emph{Theorem 5.6} hat $s$ eine Repräsentation $s(w) = \lambda\cdot\mu(w)\cdot\gamma$. Damit kann eine Repräsentation für t erzeugt werden:

\begin{align*}
    t(w) &=  \lambda\cdot\mu'(w)\cdot\gamma\\
    \mu'(w) &= \begin{cases} \mu(v)\quad\text{falls}\; w = u\#v \land v \in A^*\\
        \mu(w)\quad\text{sonst}
    \end{cases}
\end{align*}

Da $t$ die Repräsentation $(\lambda, \mu', \gamma)$ hat, ist $t$ ebenfalls nach \emph{Theorem 5.6} erkennbar.

\subsection{H 10-2}

\textsl{Sei $A = \{a,b,c\}$. Geben Sie gewichtete Automaten an, welche die folgenden Funktionen $s: A^* \rightarrow S$ über dem angegebenen Semiringen $S$ erkennen.}

\subsection{H 10-3}

\textsl{Sei $S$ ein kommutativer Semiring und $s: A^* \rightarrow S$ eine erkennbare Funktion. Für ein Wort $w = w_1 \ldots w_n \in A^*$ definieren wir $\tilde{w} = w_n \ldots w_1$. Zeigen Sie, dass die Funktion $\tilde{s} : A^* \rightarrow S$ mit $\tilde{s}(w) = s(\tilde{w})$ wieder erkennbar ist.}

\end{document}
