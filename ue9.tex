\documentclass{scrartcl}
\addtokomafont{disposition}{\rmfamily}

\usepackage{scrpage2}
\usepackage{xltxtra}
\usepackage{polyglossia}
\usepackage{lastpage}
\usepackage{amsmath, amssymb}
\usepackage{stmaryrd}
\usepackage{caption}
\usepackage{csquotes}
\usepackage{tikz}
\usepackage{fontspec}
\usepackage{enumerate}
\usepackage{mathtools} % coloneqq
% \setmainfont{Linux Libertine O}
\usetikzlibrary{arrows,automata, graphs}
\usetikzlibrary{trees}

\setmainlanguage{german}

\setlength{\parskip}{\baselineskip}
\setlength{\parindent}{0pt}

\pagestyle{scrheadings}
\clearscrheadfoot
\ihead[]{Andreas Linz, Lucas Stadler}
\chead[]{\textsc{Automatentheorie}}
\ohead[]{\today}
\cfoot[\pagemark]{\textnormal{\pagemark{} / \pageref{LastPage}}}
\setheadsepline{1pt}

\begin{document}

\section{Übung 09}

\subsection{H 9-1}

\textsl{Beweisen Sie das folgende Strukturen \emph{Semiringe} sind!}

\begin{enumerate}[(a)]
    \item $\Big(\mathcal{P}(\Sigma^*), \bigcup, \cdot, \varnothing, \{\varepsilon\} \Big)$
    \item $\Big(\mathbb{R} \cup \{\infty\}, \oplus, +, \infty, 0 \Big) \;\text{mit}\; a \oplus b = -\log\big(e^{-a}+e^{-b}\big)$
\end{enumerate}

\subsubsection{Eigenschaften eines Semiringes}
\begin{enumerate}
    \setlength\itemsep{-1em}
    \item Addition ist assoziativ
    \item Addition ist kommutativ
    \item Einselement für Addition
    \item Links- und Rechtsdistributiv für Addition und Multiplikatin
    \item Multiplikation ist assoziativ
\end{enumerate}

\subsubsection{zu (a)}

\newcommand{\baseset}{\mathcal{P}(\Sigma^*)}
Es ist klar das die Regeln für $\cup$ gelten, da die Reihenfolge der Elemente in Mengen keine Rolle spielt.
\begin{enumerate}
    \item $\forall a, b, c \in \baseset: a \cup (b \cup c) = (a \cup b) \cup c = \{a, b, c\}$
    \item $\forall a, b \in \baseset: a \cup b = \{a, b\} = b \cup a = \{b, a\}$
    \item $\forall a \in \baseset: a \cup \varnothing = \varnothing \cup a = a$
    \item \begin{align*}
        \forall a, b, c \in \baseset&:\\
            a \cdot (b \cup c) &= ab \cup ac \land\\
            (b \cup c) \cdot a &= ba \cup ca
        \end{align*}
    \item $\forall a, b \in \baseset: a \cdot (b \cdot c) = a \cdot bc = abc = (a \cdot b) \cdot c$
\end{enumerate}

\subsubsection{zu (b)}

\renewcommand{\baseset}{\mathbb{R} \cup \{\infty\}}
% \begin{enumerate}
%     \item a
%     \item a
%     \item a
%     \item a
%     \item a
% \end{enumerate}

\subsection{H 9-2}

\textsl{Geben Sie \emph{sternfreie rationale Ausdrücke} für die Sprachen an, die durch folgende First-Order Sätze beschrieben sind!}

$A = \big\{a,b,c\big\}$

Im Folgenden ist $A^* = \varnothing^c$ ein sternfreier Ausdruck, entsprechend der Definition in der Vorlesung.

\begin{enumerate}[(a)]
    \item $\forall x \forall y\Big[\big(P_b(x) \land (x < y) \land P_b(y)\big) \rightarrow \big(\forall z (x < y < z) \rightarrow \lnot P_c(z)\big)\Big]$

      Intuitiv kann man die Sprache als \enquote{zwischen zwei $b$ kommt kein $c$ vor} beschreiben.

      Ein sternfreier Ausdruck für die Sprache ist $A^* \setminus (A^* b A^* c A^* b A^*)$.

    \item $\forall x \exists y \Big( P_a(x) \rightarrow y = x+1 \land P_a(y)\Big)$

      Die Sprache kann als \enquote{nach einem $a$ muss direkt darauf folgend noch ein $a$ vorkommen} beschrieben werden. Allerdings würde dies zur Folge haben dass Wörter in denen ein $a$ vorkommt unendliche Länge haben da nach jedem $a$ noch ein $a$ vorkommen müsste.

      Da in der Vorlesung bisher keine unendlichen Wörter eingeführt wurden darf in den Wörtern der Sprache demnach kein $a$ vorkommen.

      Ein sternfreier Ausdruck für die Sprache ist deshalb $A^* \setminus (A^* a A^*)$.
    \item $\forall x \forall y \Big[ y = x+1 \rightarrow \big(P_a(x) \leftrightarrow P_b(y)\big)\Big]$

      Wenn in Wörtern der Sprache ein $a$ vorkommt und es eine direkte Nachfolgeposition gibt, so muss dort ein $b$ stehen. Umgekehrt muss wenn ein $b$ vorkommt und eine direkte Vorgängerposition gibt dort ein $a$ stehen. D.h. $a$ und $b$ kommen in $ab$-Paaren vor, wobei es einige Ausnahmen für den Anfang und das Ende von Wörtern der Sprache gibt.

      Zum Beispiel sind $b$, $bc$, $bcb$, $ab$ und $bccaba$ Wörter der Sprache.

      Ein (nicht sternfreier) Ausdruck für die Sprache ist $\{b, \epsilon\} \cdot \{ab,c\}^* \cdot \{\epsilon, a\}$.

      Ein sternfreier Ausdruck für die Sprache kann in Anlehnung zu dem sternfreien Ausdruck für $\{ab\}^*$ aus der Vorlesung konstruiert werden:

      \begin{equation*}
        \{b, \epsilon\} \cdot (\{a,b\}^* \cup \{c\}^*) \cdot \{\epsilon, a\}
      \end{equation*}

      Wobei zusätzlich die folgenden Abkürzungen verwendet werden:

      \begin{align*}
        \{a,b\}^* &= \{\epsilon\} \cup (aA^* \cap A^*b) \setminus A^* \cdot \{a^2, b^2, c\} \cdot A^*\\
        \{c\}^* &= A^* \setminus (A^* \cdot \{a, b\} A^*)
      \end{align*}
\end{enumerate}

\subsection{H 9-3}

\textsl{Bestimmen Sie die Verhalten des folgenden gewichteten Automaten über dem \emph{Semiring der rationalen Zahlen}. Geben Sie einen exakten Nachweis des bestimmten Verhaltens an!}

\end{document}
