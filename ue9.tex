\documentclass{scrartcl}
\addtokomafont{disposition}{\rmfamily}

\usepackage{scrpage2}
\usepackage{xltxtra}
\usepackage{polyglossia}
\usepackage{lastpage}
\usepackage{amsmath, amssymb}
\usepackage{stmaryrd}
\usepackage{caption}
\usepackage{csquotes}
\usepackage{tikz}
\usepackage{fontspec}
\usepackage{enumerate}
\usepackage{mathtools} % coloneqq
% \setmainfont{Linux Libertine O}
\usetikzlibrary{arrows,automata, graphs}
\usetikzlibrary{trees}

\setmainlanguage{german}

\setlength{\parskip}{\baselineskip}
\setlength{\parindent}{0pt}

\pagestyle{scrheadings}
\clearscrheadfoot
\ihead[]{Andreas Linz, Lucas Stadler}
\chead[]{\textsc{Automatentheorie}}
\ohead[]{\today}
\cfoot[\pagemark]{\textnormal{\pagemark{} / \pageref{LastPage}}}
\setheadsepline{1pt}

\begin{document}

\section{Übung 09}

\subsection{H 9-1}

\textsl{Beweisen Sie das folgende Strukturen \emph{Semiringe} sind!}

\begin{enumerate}[(a)]
    \item $\Big(\mathcal{P}(\Sigma^*), \bigcup, \cdot, \varnothing, \{\varepsilon\} \Big)$
    \item $\Big(\mathbb{R} \cup \{\infty\}, \oplus, +, \infty, 0 \Big) \;\text{mit}\; a \oplus b = -\log\big(e^{-a}+e^{-b}\big)$
\end{enumerate}

\subsubsection{Eigenschaften eines Semiringes}
\begin{enumerate}
    \setlength\itemsep{-1em}
    \item Addition ist assoziativ
    \item Addition ist kommutativ
    \item Einselement für Addition
    \item Links- und Rechtsdistributiv für Addition und Multiplikatin
    \item Multiplikation ist assoziativ
\end{enumerate}

\subsubsection{zu (a)}

\newcommand{\baseset}{\mathcal{P}(\Sigma^*)}
Es ist klar das die Regeln für $\cup$ gelten, da die Reihenfolge der Elemente in Mengen keine Rolle spielt.
\begin{enumerate}
    \item $\forall A, B, C \in \baseset: A \cup (B \cup C) = (A \cup B) \cup C = \{A, B, C\}$
    \item $\forall A, B \in \baseset: A \cup B = \{A, B\} = B \cup A = \{B, A\}$
    \item $\forall A \in \baseset: A \cup \varnothing = \varnothing \cup A = A$
    \item
    \item $\forall A, B \in \baseset: A \cdot (B \cdot C) = A \cdot BC = ABC = (A \cdot B) \cdot C$
\end{enumerate}

\subsubsection{zu (b)}

\subsection{H 9-2}

\textsl{Geben Sie \emph{sternfreie rationale Ausdrücke} für die Sprachen an, die durch folgende First-Order Sätze beschrieben sind!}

$\Sigma = \big\{a,b,c\big\}$

\begin{enumerate}[(a)]
    \item $\forall x \forall y\Big[\big(P_b(x) \land (x < y) \land P_b(y)\big) \rightarrow \big(\forall z (x < y < z) \rightarrow \lnot P_c(z)\big)\Big]$
    \item $\forall x \exists y \Big( P_a(x) \rightarrow y = x+1 \land P_a(y)\Big)$
    \item $\forall x \forall y \Big[ y = x+1 \rightarrow \big(P_a(x) \leftrightarrow P_b(y)\big)\Big]$
\end{enumerate}

\subsection{H 9-3}

\textsl{Bestimmen Sie die Verhalten des folgenden gewichteten Automaten über dem \emph{Semiring der rationalen Zahlen}. Geben Sie einen exakten Nachweis des bestimmten Verhaltens an!}

\end{document}
