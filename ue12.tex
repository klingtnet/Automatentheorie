\documentclass{scrartcl}
\addtokomafont{disposition}{\rmfamily}

\usepackage{scrpage2}
\usepackage{xltxtra}
\usepackage{polyglossia}
\usepackage{lastpage}
\usepackage{amsmath, amssymb}
\usepackage{stmaryrd}
\usepackage{caption}
\usepackage{csquotes}
\usepackage{tikz}
\usepackage{fontspec}
\usepackage{enumerate}
\usepackage{bbm}
\usepackage{mathtools} % coloneqq
% \setmainfont{Linux Libertine O}

\setmainlanguage{german}

\setlength{\parskip}{\baselineskip}
\setlength{\parindent}{0pt}

\pagestyle{scrheadings}
\clearscrheadfoot
\ihead[]{Andreas Linz, Lucas Stadler}
\chead[]{\textsc{Automatentheorie}}
\ohead[]{\today}
\cfoot[\pagemark]{\textnormal{\pagemark{} / \pageref{LastPage}}}
\setheadsepline{1pt}

\begin{document}

\section{Übung 12}

\subsection{H 12-1}

\textsl{Zeigen Sie die folgende Aussage: Sei $S$ ein kommutative Semiring und $s, t \in S^{\text{rec}}\langle\langle A^* \rangle\rangle$. Dann ist $s \varodot t$ wieder erkennbar.}

\subsection{H 12-2}

\textsl{Sei $A = \{a, b, c\}$. Geben Sie rationale Reiehen, d.h. aus Monomen und den Operationen $+$, $\cdot$ und ${}^*$ aufgebaut, an, welche die folgenden Reihen über den angebenen Semiringen $S$ beschreiben:}

In der nachfolgenden Aufgabe werden folgende Abkürzungen verwendet:

\begin{align*}
  \mathbbm{1} = (1.*)^*\\
  1.* = \sum\limits_{a \in A} 1.a\\
\end{align*}

\begin{enumerate}[(a)]
    \item $S = (\mathbb{Z}, +, \cdot, 0, 1)$, mit $s(w) = 2|w|_a -3|w|_b + 4|w|_c$

      $1.a$ beschreibt $a$, mit $\mathbbm{1} \cdot 1.a \cdot \mathbbm{1}$ wird $|w|_a$ beschrieben.

      $\mathbbm{1} \cdot \left(2.a + \left(-3\right).b + 4.c\right) \cdot \mathbbm{1}$ beschreibt $s$.

    \item $S = (\mathcal{P}(A^*), \cup, \cdot, \varnothing, \{\varepsilon\})$ mit
      \begin{equation*}
        s(w) = \begin{cases}
          \{a^n\}     &\text{falls}\;w = a^nbv\;\text{mit}\;v \in A^*\\
          \{a^2n\}    &\text{falls}\;w=a^ncv\;\text{mit}\;v \in A^*\\
          \varnothing &\text{sonst}
        \end{cases}
      \end{equation*}

      $(\{a\}.a)^+$ beschreibt $a^n$, $(\{aa\}.a)^+$ beschreibt $a^{2n}$.

      $\left(\left(\{a\}.a\right)^+ \cdot \left(1.b\right) \cdot 1\right) \cup \left(\left(\{aa\}.a\right)^+ \cdot \left(1.c\right) \cdot 1\right)$ beschreibt s.

    \item $S = (\mathbb{N}, +, \cdot, 0, 1)$ mit $(s, w) = |w|^2$
\end{enumerate}

\subsection{H 12-3}

\textsl{Sei $S = (\mathbb{Z}, +, \cdot, 0, 1)$ und $A = \{a\}$. Beweisen Sie die folgenden Aussagen:}

\begin{enumerate}[(a)]
    \item \textsl{Die Reihe $s \in S\langle\langle A^* \rangle\rangle$, welche durch $(s, a^n) = \{a^nba^n\}$ definiert ist, ist nicht erkennbar.}

    \item \textsl{Die erkennbaren Reihen über $S$ sind nicht unter dem Hadamard-Produkt $\varodot$ abgeschlossen.}

      Sei $(s, a^n) = \{a^n\}$ und $(t, a^n) = \{ba^n\}$. Es gilt $s, t \in S^{\text{rec}}\langle\langle A^* \rangle\rangle$.

      Allerdings ist $(s\,\varodot\,t, w) = (s, w) \cdot (t, w) = \{a^nba^n\}$ und damit nach (a) nicht erkennbar.
\end{enumerate}

\end{document}
