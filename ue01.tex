\documentclass{scrartcl}

\usepackage{scrpage2}
\usepackage{xltxtra}
\usepackage{polyglossia}
\usepackage{lastpage}
\usepackage{amsmath, amssymb}
\usepackage{caption}
\usepackage{csquotes}
\usepackage{tikz}
\usetikzlibrary{arrows,automata}
\usetikzlibrary{trees}

\setmainlanguage{german}

\setlength{\parskip}{\baselineskip}
\setlength{\parindent}{0pt}

\pagestyle{scrheadings}
\clearscrheadfoot
\ihead[]{Andreas Linz}
\chead[]{\textsc{Automatentheorie}}
\ohead[]{\today}
\cfoot[\pagemark]{\textnormal{\pagemark{} / \pageref{LastPage}}}
\setheadsepline{1pt}

\begin{document}

\section{Übung 01}

\subsection{H 1-1}

Die Semigruppe aller Wörter über dem endlichen Alphabet $A$ ist genau dann mit der Konkatenation $\cdot$ kommutativ, wenn $A=\{u\;|\;v = u^k,\;k \in \mathbb{N}_0\}$, also d.h. wenn $v$ eine Potenz von $u$ ist.

\begin{itemize}
    \item[$n=0$]
    \begin{align*}
        u &= A^*\\
        v&=u^n = u^0 = \epsilon\\
        u\cdot\epsilon &= \epsilon\cdot u
    \end{align*}
    \item[$n>0$]
    \begin{align*}
        v &= u^n = \overbrace{u\ldots u}^{n-mal}\\
        v \cdot u &= u^{n+1} = \underbrace{u^n}_{=v} \cdot u = u \cdot v = u \cdot u^n
    \end{align*}
    \hfill$\blacksquare$
\end{itemize}

\subsection{H 1-2}

Anmerkung: Die Zustände der Automaten wurden im Uhrzeigersinn markiert.

\subsubsection{Sprache $\{a,b\}^*ab$}

\begin{align*}
  \text{\enquote{$\subseteq$}}&: w \in L(\mathcal{A}) \Rightarrow \exists \text{ run } u \text{ für } w \text{ in } \mathcal{A}\\
  &\Rightarrow q_0 = 1, q_n = 3\\
  &\Rightarrow \text{ 3 mögliche Pfade die jeweils $ab$ ausführen: }\\
    &\hspace{3em}1 \xrightarrow{a} 2 \xrightarrow{b} 3\\
    &\hspace{3em}2 \xrightarrow{a} 2 \xrightarrow{b} 3\\
    &\hspace{3em}3 \xrightarrow{a} 2 \xrightarrow{b} 3\\
  &\text{$\mathcal{A}_1$ ist vollständig und zusammenhängend\footnotemark}\\
  &\Rightarrow u = \underbrace{q_0 \cdot\cdot\cdot q_{n-2}}_{v \in A^* = \{a, b\}^*} \xrightarrow{a} q_{n-1} \xrightarrow{b} q_n\\
  \\
  \text{\enquote{$\supseteq$}}&:\ w = w' ab \text{ mit } w' \in A^*\\
  &\Rightarrow \exists u: \underbrace{1 \rightarrow \cdot\cdot\cdot q_{n-2}}_{w'} \xrightarrow{a} 2 \xrightarrow{b} 3\\
    &\hspace{2cm}\text{mit } q_{n-2} \in \{1, 2, 3\}\\
    %\text{ da $\mathcal{A}_1$ vollständig und zusammenhängend}\\
  &\Rightarrow \text{$u$ ist ein erfolgreicher run}\\
  &\Rightarrow w \in L(\mathcal{A}_1)\\
\end{align*}
\hfill$\blacksquare$
\footnotetext{$\forall_{j,h} \in Q\;:\;\exists run (u)\;:\;q_0=j\wedge q_n=k$}

\subsubsection{Sprache $ab*$}

\begin{align*}
    \text{\enquote{$\subseteq$}}&: w \in L(\mathcal{A}_2) \Rightarrow \exists \text{\;run $u$ für $w$ in $A$}\\&\Rightarrow q_0 = 1, q_n = 3\\
    &\Rightarrow \nexists \text{ run mit } q_i = 2 \text{ für } 0 \leq i \leq n\\
    &\Rightarrow \text{einziger erfolgreicher run ist:}\\
    &\hspace{3em} 1 \xrightarrow{a} 3 \xrightarrow{b} \cdot\cdot\cdot \xrightarrow{b} 3\\
    &\Rightarrow w = ab^*\\
    \\
    \text{\enquote{$\supseteq$}} &:\ w = ab^*\\
    &\Rightarrow u: 1 \xrightarrow{a} \underbrace{3 \xrightarrow{b} \cdot\cdot\cdot \xrightarrow{b} 3}_{b^*}\\
    &\Rightarrow \text{$u$ ist erfolgreicher run für $w$}\\
    &\Rightarrow w \in L(\mathcal{A}_2)\\
\end{align*}
\hfill$\blacksquare$

\subsection{H 1-3}

$\mathcal{A} = \{w\text{ so dass }|w|_a\;\mod(2)=1, |w|_b=2\}$

\newcommand{\odd}{\mathcal{A}_{odd}}
\newcommand{\even}{\mathcal{A}_{even}}

Die Konstruktion des Automaten wurde mit Hilfe von zwei Teilautomaten durchgeführt $\even$, $\odd$, die gerade und ungerade Anzahlen von Buchstaben akzeptieren:

\begin{figure}[ht]
\centering
\begin{tikzpicture}[node distance=2cm, auto, >=stealth]
  \node[initial,state] (q_0)  {};
  \node[state,accepting by arrow] (q_1) [right of=q_0] {};

  \path[->] (q_0) edge node {a} (q_1)
            (q_1) edge [bend left] node {a} (q_0);
\end{tikzpicture}
\caption*{$\odd{}$}
\end{figure}

\begin{figure}[ht]
\centering
\begin{tikzpicture}[node distance=2cm, auto, >=stealth]
  \node[initial,state] (q_0)  {};
  \node[state] (q_1) [right of=q_0] {};
  \node[state,accepting by arrow] (q_2) [right of=q_1] {};

  \path[->] (q_0) edge node {a} (q_1)
            (q_1) edge node {a} (q_2)
            (q_2) edge [bend left] node {a} (q_1);
\end{tikzpicture}
\caption*{$\even{}$}
\end{figure}

Um insgesamt eine ungerade Anzahl an \enquote{a}s zu erhalten gibt es nun 11 verschiedene Varianten sie zwischen den \enquote{b}s zu verteilen
\begin{itemize}
    \item $o$ steht für eine ungerade und $e$ für eine gerade Anzahl an \enquote{a}s
    \item $a^*ba^*ba^*$ ($ooo$, $oee$, $eeo$, $eoe$)
    \item $a^*ba^*b$ und $ba^*ba^*$ ($oe$, $eo$)
    \item $a^*bb$, $bba^*$ und $ba^*b$ ($o$)
\end{itemize}

Da durch die angegebenen Teilautomaten bekannt ist wie ungerade bzw. gerade Anzahlen von \enquote{a}s erkannt werden können und zwischen diesen nur ein $b$ vorkommen kann, ist es möglich die 11 Varianten mit einem Automaten zu realisieren.

\begin{figure}[ht]
\centering
\begin{tikzpicture}[grow=right, sloped, ->, >=stealth]
\tikzstyle{level 1}=[level distance=2cm, sibling distance=4cm]
\tikzstyle{level 2}=[level distance=1.5cm, sibling distance=1cm]
\tikzstyle{final}=[text=red]
\node {}
child {
    node {$\odd$}
    child {node {$b$}
        child {
            node[final] {$b$}
            child {
                node[final] {$\even$}
            }
        }
        child {
            node {$\even$}
            child {
                node[final] {$b$}
                child {
                    node[final] {$\even$}
                }
            }
        }
        child {
            node {$\odd$}
            child {
                node[final] {$b$}
                child {
                    node[final] {$\odd$}
                }
            }
        }
    }
}
child {
    node {$\even$}
    child {
        node {$b$}
        child {
            node {$b$}
            child {
                node[final] {$\odd$}
            }
        }
        child {
            node {$\even$}
            child {
                node {$b$}
                child {
                    node[final] {$\odd$}
                }
            }
        }
        child {
            node {$\odd$}
            child {
                node[final] {$b$}
                child {
                    node[final] {$\even$}
                }
            }
        }
    }
}
child {
    node {$b$}
    child {
        node {$\odd$}
        child {node[final] {$b$}
            child {node[final] {$\even$}}
        }
    }
    child {node {$\even$}
        child {node {$b$}
            child {node[final] {$\odd$}}
        }
    }
    child {node {$b$}
        child {node[final] {$\odd$}}
    }
};
\end{tikzpicture}
\caption*{Spezifikation von $\mathcal{A}$, Finalzustände sind \textcolor{red}{rot}}
\end{figure}

Der Automat $\mathcal{A}$ ist nicht vollständig und aufgrund der Wahl zwischen $\odd$ und $\even$ \enquote{a}-Sequenzen auch \emph{nicht} deterministisch.

\end{document}