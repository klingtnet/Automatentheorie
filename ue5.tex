\documentclass{scrartcl}
\addtokomafont{disposition}{\rmfamily}

\usepackage{scrpage2}
\usepackage{xltxtra}
\usepackage{polyglossia}
\usepackage{lastpage}
\usepackage{amsmath, amssymb}
\usepackage{stmaryrd}
\usepackage{caption}
\usepackage{csquotes}
\usepackage{tikz}
\usepackage{fontspec}
\usepackage{enumerate}
% \setmainfont{Linux Libertine O}
\usetikzlibrary{arrows,automata}
\usetikzlibrary{trees}

\setmainlanguage{german}

\setlength{\parskip}{\baselineskip}
\setlength{\parindent}{0pt}

\pagestyle{scrheadings}
\clearscrheadfoot
\ihead[]{Andreas Linz, Lucas Stadler}
\chead[]{\textsc{Automatentheorie}}
\ohead[]{\today}
\cfoot[\pagemark]{\textnormal{\pagemark{} / \pageref{LastPage}}}
\setheadsepline{1pt}

\begin{document}

\section{Übung 05}

\subsection{H 5-1}

Alle Transitionen von $q_0$ zu $q_3$ addieren mindestens $3$, weshalb die kleinste vom Automaten akzeptierte Zahl $3$ ist. Ob die akzeptierte Zahl gerade oder ungerade ist, hängt davon ab mit welcher Zahl man $q_0$ \enquote{verlässt}. Der ursprüngliche Automat kann vom Finalzustand aus nur gerade Zahlen hinzuaddieren, akzeptiert aber dennoch, je nach Zahl aus $q_0$, alle Zahlen aus $ n \in \mathbb{N}\;|\;n \geq 3$, weshalb über $q_1$ im minimalen Automaten beliebige Zahlen aus $\mathbb{N}$ addiert werden können.

\begin{figure}[h!]
\centering
\begin{tikzpicture}[node distance=3cm, auto, >=stealth]
  \node[state, initial by arrow, initial text={}] (q_0) {0};
  \node[state,accepting by arrow] (q_1) [right of=q_0] {1};

  \path[->] (q_0) edge [loop above] node {$\mathbb{N}$} (q_0)
            (q_0) edge node {3} (q_1)
            (q_1) edge [loop above] node {$\mathbb{N}$} (q_1)
  ;
\end{tikzpicture}
\caption*{$L=\{n \in \mathbb{N}\;|\;n \geq 3\}$}
\end{figure}

\subsection{H 5-3}

Wie aus Kapitel 3 bereits bekannt ist $A^* = \varnothing^c$.

\begin{enumerate}[(a)]
  \item \begin{align*}
    a^+c^+ &= aa^*cc^*\\
          &= a\{b,c\}^c c\{a,b\}^c\\
        \{b,c\}^c  &= A^*\setminus \, A^* \cdot \{b,c\} \cdot A^* = a^*\\
        \{a,b\}^c  &= A^*\setminus \, A^* \cdot \{a,b\} \cdot A^* = c^*
  \end{align*}
  \item \begin{align*}
    (abc)^* &= \{\varepsilon\} \cup \left( aA^* \cap A^*c \right) \setminus A^* \cdot \left\{ aa, ac, ba, bb, cb, cc\right\} \cdot A^*
  \end{align*}
  \item \begin{align*}
    \left\{w\;:\;|w|_a \leq 3 \right\} &= A^* \setminus A^*\cdot\{a\}\cdot A^*\cdot\{a\}\cdot A^*\cdot\{a\}\cdot A^*\cdot\{a\}\cdot A^*
  \end{align*}
  \item \begin{align*}
    \left\{w\;:\;|w|_{abc}\right\}\,=\;&A^* \cdot \{abc\} \cdot A^* \cdot \{abc\} \cdot A^*\,\setminus\\
    & A^* \cdot \{abc\} \cdot A^* \cdot \{abc\} \cdot A^* \cdot \{abc\}  \cdot A^*
  \end{align*}
\end{enumerate}

\end{document}
