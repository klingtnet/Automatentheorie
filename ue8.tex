\documentclass{scrartcl}
\addtokomafont{disposition}{\rmfamily}

\usepackage{scrpage2}
\usepackage{xltxtra}
\usepackage{polyglossia}
\usepackage{lastpage}
\usepackage{amsmath, amssymb}
\usepackage{stmaryrd}
\usepackage{caption}
\usepackage{csquotes}
\usepackage{tikz}
\usepackage{fontspec}
\usepackage{enumerate}
\usepackage{mathtools} % coloneqq
% \setmainfont{Linux Libertine O}
\usetikzlibrary{arrows,automata, graphs}
\usetikzlibrary{trees}

\setmainlanguage{german}

\setlength{\parskip}{\baselineskip}
\setlength{\parindent}{0pt}

\pagestyle{scrheadings}
\clearscrheadfoot
\ihead[]{Andreas Linz, Lucas Stadler}
\chead[]{\textsc{Automatentheorie}}
\ohead[]{\today}
\cfoot[\pagemark]{\textnormal{\pagemark{} / \pageref{LastPage}}}
\setheadsepline{1pt}

\begin{document}

\section{Übung 08}

\subsection{H 8-1}

\textsl{Konstruieren eines MSO-Satzes nach Theorem 4.4 für den gegebenen Automaten.}

\begin{align*}
L = \exists Y_0, Y_1, Y_2 : &\bigwedge_{i+j} \lnot \exists y : \left( y \in Y_i \land y \in Y_j \right)\\
    &\land \forall x \forall y : y = x + 1 \rightarrow \Big((x \in  Y_0 \land P_a(y) \land y \in Y_1)\\
        &\qquad \land (x \in Y_0 \land P_b(y) \land y \in Y_1)\\
        &\qquad \land (x \in Y_1 \land P_a(y) \land y \in Y_0)\\
        &\qquad \land (x \in Y_1 \land P_b(y) \land y \in Y_2) \Big)\\
    &\land \exists x \forall y : \Big( x \leq y \land \big( P_a(x) \land x \in Y_1 \big) \lor \big(P_b(x) \land x \in  Y_1 \big) \Big)\\
    &\land \exists z \forall y : \big(\;y \leq z \land z \in Y_2 \;\big)
\end{align*}

Basierend auf dem Beweis für Wörter gerader Länge, da alle Wörter aus $L$ auch gerade Länge haben und zusätzlich noch ein $b$ an letzter Position.

\begin{align*}
\exists X \exists Y &: X \cup Y = \varnothing \land \forall z \big( z \in X \lor z \in Y \big)\\
    &\land \exists u \exists v \Big( (u \in X ) \land ( v \in Y ) \land \forall z : u \leq z \leq v\\
    &\qquad \land P_b(v) \land \big( \forall y \in Y : y \neq v \rightarrow P_a(y) \big) \Big)\\
    &\land \forall x \forall y  : \big( y = x + 1 \big) \rightarrow \big( x \in X \leftrightarrow y \in Y \big)
\end{align*}

\subsection{H 8-2}

$\psi : \big[ A \rightarrow [ B \rightarrow C ] \big]  \longrightarrow \big[ ( A \times B ) \rightarrow C \big]$

$f \mapsto \overline{f} : A \times B \rightarrow C $,\quad$\overline{f}(a, b) \coloneqq f(a)(b)$

\textsl{Zeigen Sie das $\psi$ bijektiv ist!}

\subsection{H 8-3}

\textsl{Für wieviele Wörter $v \in \big(A_\mathcal{V}\big)^5$ gibt es eine Belegung $\sigma$ so dass $v$ zu $(acacb, \sigma)$ korrespondiert?}

\begin{align*}
A &= \big\{ a,b,c \big\}\\
\mathcal{V} &= \big\{ x, y, X \big\}\\
x,y &\in V_{xy} = \left\{\begin{pmatrix}1\\0\\0\\0\\0\end{pmatrix}, \begin{pmatrix}0\\1\\0\\0\\0\end{pmatrix}, \ldots, \begin{pmatrix}0\\0\\0\\0\\1\end{pmatrix} \right\}\\
X &\in \mathcal{P}(X),\quad\text{card}(\mathcal{P}(X)) = 2^{\text{card}(X)}
\end{align*}

Deshalb gibt es für $\text{card}(V_{xy})^2 * \text{card}(\mathcal{P}(X)) = 5^2 * 2^5 = 800$
Wörter $v$ eine zu $(\text{acacb}, \sigma)$ korrespondierende Belegung.

\end{document}
